\documentclass[aps,pre,preprint,superscriptaddress,nofootinbib]{revtex4-1} 
%\documentclass[12pt]{article}
%\usepackage[pdftex]{graphicx}	% Include figure files
\usepackage{amssymb}
\usepackage{amsmath}
\usepackage[version=3]{mhchem}
\usepackage{natbib}

\usepackage{color}

\begin{document}

% Current Draft: 20:00 EST, 17.July.2014

\title{Synthesis of Colloidal Bowls}
% \draft command makes pacs numbers print
% repeat the \author\address pair as needed
\author{Kazem~V.~Edmond}
\email{kedmond@nyu.edu}
\affiliation{Center for Soft Matter Research, Department of Physics, New York University, New York, NY 10003, USA}
\author{Tess~W.~P.~Jacobson}
\affiliation{Center for Soft Matter Research, Department of Physics, New York University, New York, NY 10003, USA}

% Professors
\author{Gi-Ra~Yi}
\affiliation{Department of Polymer Science and Engineering, Sungkyunkwan University, Suwon 440 746, Republic of Korea}
\author{Andrew~H.~Hollingsworth}
\affiliation{Center for Soft Matter Research, Department of Physics, New York University, New York, NY 10003, USA}
\author{Stefano~Sacanna}
\altaffiliation{Center for Soft Matter Research, Department of Physics, New York University, New York, NY 10003, USA}
\altaffiliation{Current address:
Molecular Design Institute, Department of Chemistry, New York University, New York, NY 10003, USA}
\author{David~J.~Pine}
\email{pine@nyu.edu}
\affiliation{Center for Soft Matter Research, Department of Physics, New York University, New York, NY 10003, USA}


\begin{abstract}

We describe a general procedure for fabricating bowl-shaped colloidal particles using an emulsion templating technique.
We prepare a colloidal solution of biphasic particles by heterogeneous nucleation of oil droplets in the presence of solid seed particles.
The seeds, which each sit at a droplet interface, are plasticized by adding a solvent.
The liquefied seed phase is deformed by surface tension, using the droplet surfaces as a template.
Evaporating the solvent re-solidifies the seed particles.
We dissolve and remove the oil phase by transferring the aqueous colloid to alcohol, leaving behind the bowl-shaped particles.
The particles are studied using scanning electron microscopy and optical microscopy.
Particle curvature and diameter can be carefully tuned by adjusting droplet size and the size of the seed particles.
The colloiid's uniform size and curvature allows for the formation of flexible colloidal chains and clusters via the depletion interaction.

\end{abstract}
\pacs{XXXXXXXXX}
%%% ----------------------------------------------------------------------
\maketitle

%%%%%%%%%%%%%%%%%%%

\section{Introduction}

% I need an introduction.  A justification for making these things.

% Very rough draft:

Colloidal particles with unique non-spherical shapes are useful as models of existing complex molecular materials and also as building blocks for the self-assembly of new meta-materials \cite{Marechal2010Phase, Marechal2010Phase2, Edmond2012Decoupling, Sacanna2013Engineering}.
Both applications require the ability to fabricate bulk quantities of material while maintaining precise control over particle morphology, requirements that are usually at odds with one another.

Bowl-shaped colloidal particles have recently gathered significant interest among colloidal scientists.
Packings of these particles exhibit a rich variety of unique structural configurations and phase behaviors, recently observed in both simulations and experiments, and serve as models of ferroelectric fluids and exotic liquid crystals \cite{Marechal2010Phase, Marechal2010Phase2, Zoldesi2006Deformable, Cinacchi2010Phase, Cinacchi2013Phase}.
Alternatively, a bowl-shaped particle's concavity provides a ``lock and key'' mechanism, providing a means of self-assembling colloidal buildings for meta materials \cite{Sacanna2010Lock, Sacanna2013Shaping, Sacanna2013Engineering}.
There are a number of ways to produce bowl-shaped particles \cite{Zoldesi2006Deformable, Sacanna2013Shaping, Wang2002Seeded}, but as of yet a means of scaling up production while still retaining precise control over particle morphology has yet to be discussed.

In this manuscript we describe a means for the bulk fabrication bowl-shaped colloidal particles.
Our method is based on a general emulsion templating strategy recently described by Sacanna, et al. \cite{Sacanna2013Shaping}.
We demonstrate our method's scalability through the production of many grams of monodisdperse colloidal particles.
Despite producing huge numbers of particles, we are able to precisely vary particle size and shape with ease.
Our technique is highly versatile as it works with a variety of different chemicals and materials.
Additionally, we demonstrate the utility of these particles by assembling, via the depletion interaction, colloidal chains and also reconfigurable colloidal clusters.
% The reconfigurable cluster was first thought of by GiRa, and made by Tess and I.

% We make bowl-shaped particles which have concavities, like lock and key colloid that are great for self-assembly.
% But also, columnar phases of bowl-shaped things are super interesting to a number of people.
% Molecular liquid crystals, for example, might be improved by bowl-shaped molecules.
% We make shitloads of lenses that are all monodisperse, and with carefully controlled shapes.
% We demonstrate their usefulness as a model system and also for self-assembly by showing chains and reconfigurable clusters.  And it's great.

%Colloidal particles with concavities, or so called ``lock and key'' particles, also hold promise as building-blocks for the self-assembly of building blocks for novel meta-materials \cite{Sacanna2010Lock, Sacanna2013Shaping, Sacanna2013Engineering}.

\section{Results}

\subsection{Synthesis of colloidal bowls}

Monodisperse charge-stabilized bowl-shaped particles are fabricated from sulfonated polystyrene (PS) microspheres using a variation of a procedure first described by Sacanna, et al. \cite{Sacanna2013Shaping}.
The technique involves three general steps.
First, aqueous suspensions of biphasic particles are prepared by heterogeneous nucleation of oil droplets in the presence of solid PS seeds.
The seed particles, assuming they are monodisperse, will each have an equal volume of oil droplets wetting their surface.
Due to their charge and wetting, the seeds sit at the droplets' oil–water interface.
The oil droplets consist of 3-methacryloxypropyl trimethoxysilane (TPM) oligomers that are generated in situ via a base-catalysed condensation reaction between hydrolysed TPM monomers.
Second, we plasticize the seed particles with a solvent.
The liquefied PS does not mix with the oil phase but is instead deformed by surface tension, conforming to the shape of an oil droplet, using it as a template.
The third and final step is to remove the solvent, by evaporation, and to then dissolve and remove the oil phase by transferring the colloid into alcohol, leaving behind the solid bowl-shaped particles.
At this point the particles are transferred back into pure water where, in the presence of a depletant, the particles assemble chains.
In the presence of spherical particles and a depletion force, the particles will assemble onto the particle surfaces producing reconfigurable colloidal clusters.

% Fig 1: A cartoon schematic of each step, with complimentary microscope images.

\subsubsection{Different sizes}

Our procedure's general nature accommodates a range of particle sizes.
Figure~\ref{sizes} contains scanning electron micrographs of particles fabricated from seeds of different sizes, (a) 200~nm, (b) 500~nm, and (c) 1~$\mu$m in diameter.

In this type of fabrication procedure, where we are increasing the size of emulsion droplets, sedimentation can become problematic for maintaining a monodisperse distribution of particle sizes.
As the droplets grow on the particle surfaces, their rate of diffusion decreases and their rate of sedimentation increases.
The sample must be stirred to continuously disperse the particles but without shearing the droplets, which can either break them up or induce coalescence.
For smaller sample volumes we use an Erlenmeyer flask that lets us easily periodically swirl the sample with minimal splashing or spilling.
For larger samples, we use a round-bottom flask and stir with an impeller, with a relatively small curved blade, positioned mid-height within the solution.
To avoid crushing the emulsion droplets, it is important that the impeller blades do not touch the bottom of the vial.


A particle's curvature, or concavity, is determined by the diameter of the nucleated oil droplet.

\subsubsection{Different curvatures}




\subsection{Detailed procedure}

Unless otherwise noted, all materials are used as received.
For the PS seeds, we use either commercially available particles (Thermo Scientific) or synthesize them ourselves using a basic surfactant-free emulsion polymerization procedure \cite{Song1990Kinetics}.
The presence of surfactant and other impurities will cause the homogeneous nucleation of secondary TPM droplets throughout the colloid.
To avoid this, thoroughly wash the seed particles in deinonized water, either by dilution in dialysis tubing or repeated sedimentation, decanting, and resuspension.



% too much detail for an abstract
Aqueous suspensions of biphasic particles are prepared by heterogeneous nucleation of oil droplets in the presence of solid polystyrene (PS) seeds. The oil consists of 3-methacryloxypropyl trimethoxysilane (TPM) oligomers that are generated in situ via a base-catalysed condensation reaction between hydrolysed TPM monomers.
Due to their charge and wetting, the seeds sit at the droplets' oil–water interface.
We plasticize the particles by introducing a solvent.
The liquefied PS does not mix with the oil phase.


\textbf{Preparation of Microspheres:}

Our bowl-shaped particles are fabricated by the plasticization and deformation of sulfonated PS microspheres, which can be purchased or synthesized.

To synthesize 500~nm microspheres, we follow a standard surfactant-free emulsion polymerization procedure \cite{Song1990Kinetics}.

In a 500~mL 3-neck reaction vessel, we add 175~g of deionized water (Millipore).
A water-jacketed condenser is attached to one of the necks for solvent refluxing.
Via a side neck, the reactor system is kept at a slightly positive nitrogen pressure.
Using a circulating bath, the reactor is set to 70~C. While stirring with an impeller at 300~rpm, we disperse 22.1~mL of uninhibited styrene monomer (Sigma Aldrich) and wait several minutes for it to thoroughly emulsify.
Once the solution's temperature has equilibrated, we disperse 0.2~g of potassium persulfate (KPS) in 5~mL of water which we rapidly inject into the solution.
After roughly 24~hours, or when all of the styrene monomer has been consumed, the dispersion of particles is thoroughly washed in deionized water using centrifugation and decantation. 

%To fabricate the bowl-shaped particles, we follow a modified version of a procedure first described by Sacanna, et al. \cite{Sacanna2013Shaping}.

\textbf{Nucleation:}

In a round-bottom flask, we disperse sulfonated PS microspheres in deionized water, producing a concentration of less than 0.5\% (v/v).
The solution's pH is increased to above 9 by adding a sufficient amount of NH3 (28 wt.\%, Sigma Aldrich).
While gently stirring with an impeller, we disperse 3-trimethoxy propyl methacrylate oil (TPM) into the solution.
Via hydrolysis and subsequent condensation reaction, TPM droplets form on the particles' surfaces.
We grow the droplets to approximately 1.5~times the particle size by gradually adding additional TPM.
Droplet size is monitored using optical microscopy.
The occurrence of secondary nucleation, where droplets form homogenously throughout the solvent phase, can be adding TPM oil at a slower rate or by reducing the solution's pH, but not below 9.

Initially, many small TPM droplets will form across each particle's surface.
As TPM oil is added the droplets will grow and coalesce with one another.
Coalescence can be induced, without continuing to grow the droplets, by adding trace amounts of solvent, such as toleune of dichloromethane, for example.
Small amounts of these solvents will not deform the particles, but will significantly alter the surface tension at the oil-water-particle interface.


\textbf{Deformation:}
Next, to deform the particles into a bowl shape, we disperse dichloromethane (DCM) into the solution. Typically, we oversaturate the solution's water content by roughly a factor of two.  We continue to stir the solution for up to an hour, until the particles appear to be fully deformed.  We gradually heat the solution to 40~C to boil off the DCM, solidifying the now bowl-shaped particles.

\textbf{Washing:}
To remove the TPM droplets, we wash the colloid in alcohol. First, we sediment and decant the colloid.  Filling only half the vial with deionized water, we fully redisperse the particles.  The remaining space is gradually filled with a solution of ethanol and polyvinylpyrrolidone (PVP, 10~wt.~\%) (Sigma Aldrich), which serves as a good surfactant for PS in alcohol.  The colloid is thoroughly stirred and gently sonicated.  Subsequently, we decant the colloid in pure ethanol at least three times.  Once the entirety of the TPM has been removed, we may transfer the particles to deionized water.  Note that during the deformation procedure, TPM oil seeps into the particles via the DCM.  Thorough removal of the TPM oil prevents particle flocculation in the aqueous phase.


\textbf{Chains:}

We direct the self-assembly of flexible colloidal chains in an aqueous solution by inducing a depletion interaction between monodisperse bowl-shaped particles.
We first disperse the particles in \ce{H2O} at a volume fraction of approximately $\phi=0.5~\%$.
We add 10~mM of sodium chloride (NaCl), 0.01~wt.\% of tetramethyl ammonia hydroxide (TMAH), and 0.05~wt.\% Pluronics F108.
NaCl screens electrostatic repulsion.
F108 serves as  a steric stabilizer for the particles.
The TMAH increases the solution's pH, increasing the cover-slip's surface charge, preventing the particles from adsorbing to the boundary's walls \cite{Sacanna2010Lock}.
For the depletion interaction we use poly ethylene oxide (PEO) with a molecular weight of 600k.
Using particles with a diameter of approximately 2~$\mu$m, we add  0.6~g/L of PEO.
We observe the formation of chains after only a few minutes.
Smaller particles, with a diameter of approximately 1~$\mu$m, require a stronger interaction with 0.7~g/L of PEO.


\textbf{Reconfigurable clusters:}

We decorate spherical particles, solid microspheres or droplets, with multiple bowl-shaped particles using a depletion interaction.
In a solution of NaCl (10 mM), TMAH (0.01~wt.\% ), and F108 (0.05~wt.\%), we disperse both colloidal microspheres and bowl-shaped particles.
Here, it is important for the colloidal bowls to have a curvature slightly less than that of the microspheres.
An exact match in curvature will require an entropic loss greater than the gain provided by depletion \cite{Sacanna2010Lock}.
% This is the famous entropic term in William's lock and key model.
With a sufficiently high number ratio of bowls to spheres, we observe the formation of particles with three or four mobile ``patches'', as shown in Fig.~\ref{GiRaCluster}.


%%%%%%%%%%%%%%%%%%%%%%%%%%%%%%%%%%%%%%%%
\section{Acknowledgements}
The work of K.V.E., T.W.P.J. and D.J.P. was supported by the National Science Foundation under Grant No. DMR-1105455. (what about Andy and Stefano?)

\bibliography{kedmond}

\end{document}